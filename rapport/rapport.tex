\documentclass[a4paper,11pt]{article}

\usepackage{lscape}
\usepackage{geometry}
\usepackage[utf8]{inputenc}
\usepackage[francais]{babel}
\usepackage[T1]{fontenc}
\usepackage{gantt}
\usepackage{tikz}
\usepackage{tikz-uml}
\usepackage{relsize}
\usepackage{color}
\definecolor{dkgreen}{rgb}{0,0.6,0}
\definecolor{gray}{rgb}{0.5,0.5,0.5}
\definecolor{mauve}{rgb}{0.58,0,0.82}

\usepackage{listings}
\usepackage{float}
%\usepackage{kpfonts}

\usepackage{graphicx}
%\usepackage{rotating}

\lstset{
  language=C++,
  basicstyle=\footnotesize,
  backgroundcolor=\color{white},
  keywordstyle=\color{red},
  commentstyle=\color{dkgreen},
  stringstyle=\color{mauve},
  numberstyle=\color{red},
  morekeywords={string},
  frame=BL,
  aboveskip=1em,
  belowskip=2em,
}
\lstset{
  literate={ù}{{\`u}}1
  {é}{{\'e}}1
  {è}{{\'e}}1
  {à}{{\`a}}1
}


\lstdefinelanguage{tikzuml}{language=[LaTeX]TeX, classoffset=0, morekeywords={umlbasiccomponent, umlprovidedinterface, umlrequiredinterface, umldelegateconnector, umlassemblyconnector, umlVHVassemblyconnector, umlHVHassemblyconnector, umlnote, umlusecase, umlactor, umlinherit, umlassoc, umlVHextend, umlinclude, umlstateinitial, umlbasicstate, umltrans, umlstatefinal, umlVHtrans, umlHVtrans, umldatabase, umlmulti, umlobject, umlfpart, umlcreatecall, umlclass, umlvirt, umlunicompo, umlimport, umlaggreg}, classoffset=1, morekeywords={umlcomponent, umlsystem, umlstate, umlseqdiag, umlcall, umlcallself, umlfragment, umlpackage}, classoffset=0,  sensitive=true, morecomment=[l]{\%}}

\geometry{margin=2.5cm}
\geometry{headheight=15pt}

\usepackage{fancyhdr}
\usepackage{fancyvrb}
\usepackage{float}
\usepackage[footnote,smaller]{acronym}

\pagestyle{fancy}
\rhead{RE203 - Projet de Réseau}

% \acrodef{LABRI}{Laboratoire Bordelais de Recherche en Informatique}

\begin{document}

\begin{titlepage}
  \begin{center}

    \textsc{\LARGE RE203 - Projet de Réseau}\\[2cm]
    \textsc{\large Rapport Final}\\[3cm]
    Maxime \textsc{Bellier} \ \ \ Jean-Michaël \textsc{Celerier}\\
    Julien \textsc{Chaumont} \ \ \ Bazire \textsc{Houssin} \ \ \ Sylvain \textsc{Vaglica}\\[1cm]
    \textsc{Groupe 3}\\[1.5cm]
    \textsc{\large 23/05/2013 }\\[1.5cm] %TODO
    \includegraphics[width=8cm]{logo.png}

  \end{center}
  \vspace{3cm}

\end{titlepage}

\clearpage

\section*{Introduction}

Ce rapport final pour le projet de réseau RE203 fait suite au rapport intermédiaire délivré le 23 avril dernier. Il sera ici question de l'implémentation mise en \oe uvre par notre groupe, l'explication du sujet ayant été développée précédemment.

\section{Le contrôleur}


\section{Les routeurs}


\end{document}
